\documentclass[a4paper,12pt]{report}

\usepackage[spanish,es-tabla]{babel}
\usepackage[utf8]{inputenc}
\usepackage{csquotes}
\usepackage{graphicx}
\usepackage{adjustbox}
\usepackage{xcolor}
\usepackage{lmodern}
\usepackage{framed}
\usepackage{lipsum}  
\usepackage{parskip}
\usepackage{listings}

\usepackage[toc,page]{appendix}
\renewcommand{\appendixtocname}{Anexos}
\renewcommand{\appendixpagename}{Anexos}


\usepackage{titlesec}
  \titleformat{\chapter}[hang]
    {\normalfont\huge\bfseries}
    {\thechapter}{20pt}{\huge}

\renewcommand{\familydefault}{\sfdefault}

\usepackage{geometry}
 \geometry{
 a4paper,
 total={170mm,257mm},
 left=20mm,
 top=20mm,
 }

 % Bibliografía
\usepackage[%
    sorting=none,
    style=ieee,
    citestyle=numeric-comp
]{biblatex}
\addbibresource{biblatex-examples.bib}

 % Glosario
\usepackage{glossaries}
\makeglossaries

\colorlet{shadecolor}{blue!5}

\newenvironment{zaguancolor}
  {\def\FrameCommand{\fboxsep=\topsep\colorbox{shadecolor}}%
  \MakeFramed {\advance\hsize-\width \FrameRestore}}%
 {\endMakeFramed}

\newenvironment{zaguan}
  {\begin{zaguancolor}}
  {\end{zaguancolor}}

\begin{document}

%%%%%%%%%%%%%%%%%%%%%%%%%%%%%%%%%%%%%%%%%%%%%%%%%%%%%%%%%%%%%%%%%%%%%%%%%%%%%%%%%%
%		PORTADA
%%%%%%%%%%%%%%%%%%%%%%%%%%%%%%%%%%%%%%%%%%%%%%%%%%%%%%%%%%%%%%%%%%%%%%%%%%%%%%%%%%

\begin{titlepage}

    \vspace*{5mm}
    \begin{figure}[!h]
        \centering
        \includegraphics[width=9.52cm]{img/unizar}
    \end{figure}

    \vspace*{10mm}

    \fontsize{28pt}{28pt}\selectfont

    \begin{center}
        \setlength{\fboxsep}{3.4mm}
        \adjustbox{minipage=16.4cm,cfbox=blue,center}{%
            \begin{center}Trabajo Fin de Grado/Máster\end{center}
        }
    \end{center}

    \vspace*{18.7mm}


    \fontsize{20pt}{20pt}\selectfont
    \begin{center} Título del trabajo y subtítulos del trabajo\end{center}
    \begin{center} Title and subtitle (if required)\end{center}

    \vspace*{1cm}
    \begin{center}
        \fontsize{12pt}{12pt}\selectfont
        \center{Autor/es}

        \vspace*{3.65mm}
        \fontsize{18pt}{18pt}\selectfont
        \center{Nombre y apellido del/los autor/es}
        \vspace*{2cm}
        \fontsize{12pt}{12pt}\selectfont
        \center{Director/es}
        \vspace*{3.56mm}
        \fontsize{14pt}{14pt}\selectfont
        \center{Nombre y apellido del/los director/es}
    \end{center}


    \vspace*{16.45mm}
    \fontsize{12pt}{12pt}\selectfont
    \begin{center}
        Facultad / Escuela\\
        Año\\
    \end{center}

    \vspace*{16.45mm}
    \begin{zaguan}
        \centering
        \textbf{Repositorio de la Universidad de Zaragoza - Zaguan http://zaguan.unizar.es}
    \end{zaguan}

    \setcounter{footnote}{1}
    \renewcommand{\thefootnote}{\arabic{footnote}}
    \pagenumbering{gobble}

\end{titlepage}

\clearpage\mbox{}\clearpage

\pagenumbering{Roman}
\setcounter{page}{2}

%%%%%%%%%%%%%%%%%%%%%%%%%%%%%%%%%%%%%%%%%%%%%%%%%%%%%%%%%%%%%%%%%%%%%%%%%%%%%%%%%%
%		AGRADECIMIENTOS
%%%%%%%%%%%%%%%%%%%%%%%%%%%%%%%%%%%%%%%%%%%%%%%%%%%%%%%%%%%%%%%%%%%%%%%%%%%%%%%%%%

\chapter*{Agradecimientos}
\vspace{1cm}
\lipsum[2-4]

%%%%%%%%%%%%%%%%%%%%%%%%%%%%%%%%%%%%%%%%%%%%%%%%%%%%%%%%%%%%%%%%%%%%%%%%%%%%%%%%%%
%		RESUMEN
%%%%%%%%%%%%%%%%%%%%%%%%%%%%%%%%%%%%%%%%%%%%%%%%%%%%%%%%%%%%%%%%%%%%%%%%%%%%%%%%%%

\chapter*{Resumen}
\vspace{1cm}
\lipsum[2-4]

%%%%%%%%%%%%%%%%%%%%%%%%%%%%%%%%%%%%%%%%%%%%%%%%%%%%%%%%%%%%%%%%%%%%%%%%%%%%%%%%%%
%		TABLA DE CONTENIDOS
%%%%%%%%%%%%%%%%%%%%%%%%%%%%%%%%%%%%%%%%%%%%%%%%%%%%%%%%%%%%%%%%%%%%%%%%%%%%%%%%%%

\tableofcontents

\newpage
\renewcommand\listtablename{Lista de Tablas}
\listoftables

\newpage
\renewcommand\listfigurename{Lista de Figuras}
\listoffigures

\newpage
\pagenumbering{arabic}
\setcounter{page}{1}


\chapter{Introducción}

Breve introducción. Aquí introducimos brevemente el TFG. 

\section{Introducción a XXX}

Introducción al contexto, al problema, a la tecnología que caracteriza al TFG.

\section{Situación de partida}

Descripción del estado inicial del problema o situación a resolver o mejorar con el TFG.

\section{Objetivos}

El objetivo principal es nuevo sistema o la solución a implementar. 
Se presenta acompañado de los subobjetivos principales, normalmente a nivel funcional.
La redacción de los objetivos y subobjetivos debe ser SMART (Specific, Measurable, Achievable, Relevant, Time-bound).
En todo caso, no debe ser una mera descripción de lo hecho. 

\section{Alcance}

Se resumen cuáles son los resultados del proyecto que se presentan en esta memoria 
y qué resultados podrían entenderse que deberían ser parte del proyecto, pero que 
no se han realizado o se ha decidido no hacer.

\section{Metodología}

Describes los pasos que ha tenido el proyecto.

Describes a continuación, 
cómo se corresponden con la organización de la memoria.

\chapter{Análisis}
\lipsum[2-4]

\chapter{Diseño}
\lipsum[2-4]

\chapter{Desarrollo}
\lipsum[2-4]

\chapter{Pruebas}
\lipsum[2-4]

\chapter{Conclusiones}
\lipsum[2-4]

\printglossary[nonumberlist,type=\acronymtype,title={Acrónimos}]

\clearpage
\glsaddall
\printglossary[nonumberlist]

\clearpage
\nocite{*}
\printbibliography[heading=bibintoc,title={Referencias}]

\begin{appendices}

\chapter{Recomendaciones generales}

\begin{itemize}
\item No hay que olvidar que el lector de la memoria no tiene por qué ser un experto en el tema.
\item La memoria se redacta después de haber finalizado el proyecto, por lo que 
no se debe hablar de lo que se va a hacer sino de lo que se ha hecho.
\item El uso del tiempo verbal futuro debe ser mínimo.
\end{itemize}

\chapter{Cómo citar}

El fichero \texttt{biblatex-exmples.bib} contiene muchos ejemplos de citas. 
Todos ellos están en la Bibliografía, ya que hemos usado el comando \LaTeX{} \texttt{\char`\\nocite\{*\}}.

A continuación vamos a ver distintos tipos de citas:
\begin{itemize}
    \item Esta es una cita simple~\cite{westfahl:space}.
    \item Esta es una cita con una pre-nota y una post-nota~\cite[pre-nota][post-nota]{westfahl:space}.
    \item Esta es una cita con una pre-nota y un número de página~\cite[pre-nota][42]{westfahl:space}.
    \item Esta es una cita con un número de página~\cite[][42]{westfahl:space}.
    \item Esta es una cita múltiple sin rango~\cites{westfahl:space, aksin}.
    \item Esta es una cita múltiple con rango~\cites{westfahl:space, angenendt, aksin}.
\end{itemize} 

\chapter{Cómo generar texto de relleno}

El paquete \texttt{lipsum} es un paquete de LaTeX diseñado para generar texto de relleno, comúnmente conocido como ``Lorem Ipsum''.
Este tipo de texto se utiliza para llenar espacios de muestra en documentos y maquetaciones cuando el contenido real aún no está disponible. 
El paquete \texttt{lipsum} proporciona comandos simples para insertar párrafos o fragmentos de texto de relleno en diferentes partes de tu documento.

El paquete \texttt{lipsum}, ya está incluido en el documento.

\begin{lstlisting}[language={[LaTeX]TeX},backgroundcolor=\color{gray!10!white}]
\usepackage{lipsum} 
\end{lstlisting}

Una vez que has incluido el paquete, puedes utilizar los siguientes comandos en tu documento para insertar texto de relleno:

\begin{itemize}
  \item \texttt{\char`\\lipsum[1-n]}: Inserta los primeros \texttt{n} párrafos de texto de relleno.
  \item \texttt{\char`\\lipsum[n-m]}: Inserta los párrafos del \texttt{n} al \texttt{m} de texto de relleno.
  \item \texttt{\char`\\lipsum}: Inserta un rango de párrafos, por defecto del párrafo 1 al 7 (equivale a \texttt{\char`\\lipsum[1-7]}).
\end{itemize}

A continuación, te muestro un ejemplo básico de cómo puedes usar el paquete \texttt{lipsum} en un documento \LaTeX{}:

\begin{lstlisting}[language={[LaTeX]TeX},backgroundcolor=\color{gray!10!white}]
\lipsum[2-4] 
\end{lstlisting}
  
Este ejemplo generará el siguiente párrafo:

\lipsum[2-4]

\end{appendices}

\end{document}